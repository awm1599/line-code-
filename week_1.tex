\documentclass[UTF-8]{article}
\usepackage{ctex}
\usepackage{CJK}
\usepackage{amsmath}
\usepackage{graphicx}

\begin{document}
	\title{\heiti 学习报告 Week 1}\quad
	\author{\kaishu 韩啟泰}\quad
	\date{2019年7月14日}
	\maketitle
	{\songti }
	\section{机器学习(machine learning)}
		机器学习发源于人工智能领域,我们希望创造出具有智慧的机器,通过编程来让机器完成一些基础的工作。如如何寻找A到B的最短路径,网页搜索照片标签或邮件反垃圾系统,做这些事情的唯一方法是让机器自己学习自己去做,因此机器学习已经发展成为计算机的一项新能力,并且与工业界和基础学科界有着紧密的联系
	\section{数据挖掘(database mining)}
		机器学习如此普遍的原因之一就是网络和自动化技术的快速发展,我们拥有大量的数据集,并试图采用机器学习算法来挖掘数据,来更好的理解用户,并更好的为用户服务。如医疗记录,假如能将医疗记录转化为医疗知识,那就能更好的理解疾病,此外还有计算生物学、工程……
	\section{监督学习(回归问题)}
		预测一个连续值的输出,比如房价、肿瘤预测等等,实际上还是一个离散值,但通常把它看作实际数字,是一个标量值,一个连续的数。而回归,意味着要预测这类连续值属性的种类
		\subsection{房价预测}
		\subsubsection{预测函数(假设函数)}\\
			$h(\theta)=\theta + \theta x$\\
			对于给定的数据集,我们需要根据X合理的预测Y值,在线性回归中,还需要解决一个最小化的问题\\
			\subsubsection{代价函数(平方误差代价函数)(平方误差函数)}\\
			$J(\theta_0,\theta_1)=\dfrac{1}{2m} \sum\limits_{i=1}^{m} [h\theta(x^i)-y^i]^2$\\
			\subsubsection{目标函数}\\	
			$minimize_{(\theta_0,\theta_1)} J(\theta_0,\theta_1)$
			\subsubsection{梯度下降算法(线性回归)}
			$\theta_j:=\theta_j-\alpha\dfrac{\partial J(\theta_0,\theta_1)}{\partial \theta_j} $
			$\alpha$是一个数字,被称为学习效率
	\section{无监督学习(聚类算法)}
		没有属性或标签这一概念,所有数据都是一样的没有区别,在无监督学习中,没有人告诉我们怎么做,我们也不知道每个数据点究竟是什么意思。对于给定的数据集,无监督学习算法可以判定,该数据包含的两个不同的聚类。比如谷歌新闻的分类,组成新闻专题
		\subsection{肿瘤预测}

\end{document}



